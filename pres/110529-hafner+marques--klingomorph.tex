\documentclass[notes]{beamer}
\usepackage{beamerthemesplit}
\usepackage{graphics}
\usepackage[utf8]{inputenc}

\begin{document}

\title{Finite-State Methoden in der Sprachtechnologie
\\
``Klingomorph'':
\\
\emph{Eine Morphologie (und Morphotaktik) für das (Neu-)Klingonische}}
\author{Simon Hafner / Hernani Marques}
\date{30.05.2011}

\frame{\titlepage}

\section{Klingomorph: Goals}

\frame{

\frametitle{Goals}

\begin{itemize}
\item Erkennung von Verben und Substantiven
\item Hauptarbeit erledigt in lexc
\end{itemize}
}

\section{Klingomorph: Charakteristik des Klingonischen}

\frame{

\frametitle{Charakteristik des Klingonischen}

\begin{itemize}
\item Das Klingonische ist eine stark \emph{agglutinierende Sprache}
\item Verben kennen 9 Suffixe (jeweils optional, aber geordnet) und auch Präfixe
\item Nomen kennen 5 Suffixe (dito)
\item Bei Verben sind zudem 4 Suffixe (``rovers'') zwischen den anderen Suffixen
frei setzbar
\end{itemize}
}

\section{Klingomorph: Zum Projekt}

\frame{

\frametitle{Zum Projekt}

\begin{itemize}
\item Unser Projekt ist opensource verfügbar: \url{https://github.com/2mh/klingomorph}
\item Grossmehrheitlich ist das Projekt in lexc implementiert
\item Einige Details bzgl. den freien Verbsuffixen oder den zu wählenden Nomensuffixen
sind in \emph{foma} bzw. \emph{xfst} implementiert
\end{itemize}
}



\end{document}
